\chapter{总结与展望}
    \section{与相关工作的对比}
        目前, 对于web API, 亦有其他自动化测试方法. 本小节将本文提出的方法与学术界相关工作、工业界仍普遍采用的手工脚本测试法, 从表达控制流能力、表达数据流能力以及测试用例多样性方面, 进行了简要对比.
    
        具体来说, 考虑以下3种测试方法作为对比:
        \begin{itemize}
            \item 概率转移图\cite{junyiw17}: 此模型使用有向图表示. 在图上进行遍历, 并根据出边的权值概率性地选择下一节点, 便可以自动化地生成测试用例.
            
            \item 请求序列: 文献\cite{taox06}\cite{xiaodongg16}中, 通过静态分析和数据挖掘方法, 从开源仓库中获得API请求序列的频繁模式. 这种序列可以使用标准化格式编码, 并导入Swagger Inspector工具\cite{swaggerinspetor17}执行.
            
            \item 测试脚本: 在工业界的企业中, 最常见的仍然是使用测试人员手工编写的测试脚本.
        \end{itemize}
        
        \begin{table}[!htb]
            \centering
            \begin{tabular}{ccccc}
                \toprule
                 & 自动化程度 & 控制流 & 数据流 & 测试用例多样性 \\
                \midrule
                本文的场景 & 半自动构建 & 顺序, 循环,  & 数据复用 & 满足多种 \\
                模型方法   & 自动执行 & 条件与概率转移 & 数据依赖 & 覆盖率要求 \\
                \hline
                \multirow{2}{*}{概率转移图\cite{junyiw17}} & 半自动构建 & 顺序与 & \multirow{2}{*}{无} & 满足基于图的 \\
                & 自动执行 & 概率转移 &  & 覆盖率要求 \\
                \hline
                \multirow{2}{*}{请求序列\cite{taox06}\cite{xiaodongg16}} & 自动构建 & \multirow{2}{*}{顺序} & 数据复用 & 固定 \\
                & 自动执行 &  & 数据依赖 & 序列 \\
                \hline
                \multirow{2}{*}{测试脚本} & 手工构建 & 可完全 & 可完全 & 取决于 \\
                & 自动执行 & 编程自定义 & 编程自定义 & 具体脚本 \\
                \bottomrule
            \end{tabular}
            \caption{本文测试方法与其他常用测试方法的对比.}
            \label{tab:related_work_compare}
        \end{table}
        
        表\ref{tab:related_work_compare}对这些不同方法进行了对比. 
        \begin{itemize}
            \item 在自动化程度方面, 基于本文提出的测试方法, 测试人员只需要设计高层的场景模型即可, 不需要编写测试脚本和处理测试细节, 因此具有半自动构建的特性, 相比手工构建可有效提高测试效率; 概率转移图方法与之类似, 只需要设计概率转移图; 请求序列方法基于自动化挖掘出的频繁序列模式, 可全自动构建; 而测试脚本需完全手动编写.
            \item 在控制流方面, 本文提出的场景模型支持顺序、循环、条件与概率转移, 能力可与过程式程序语言媲美; 概率转移图方法仅支持顺序与概率转移, 请求序列方法则仅支持顺序执行, 测试脚本方法基于程序语言, 能力与程序语言的控制流表达能力等价.
            \item 在数据流方面, 本文提出的场景模型可描述数据复用与数据依赖, 与程序语言的能力近似; 概率转移图方法不支持数据流描述; 请求序列方法进行一定的扩展, 则也可以支持数据复用与数据依赖的描述; 而测试脚本基于程序语言, 描述能力与可执行程序相同.
            \item 在测试用例多样性方面, 从实验中即可看出, 本文提出的场景模型可自动生成具有相当多样性的测试用例, 从而满足多种覆盖率要求; 概率转移图方法生成的测试用例仅具有图上的路径多样性, 可满足基于图的覆盖率要求, 但不满足其他覆盖率要求如数据分区覆盖率要求; 请求序列方法只能生成固定执行序列, 测试用例几乎不具有多样性; 测试脚本的用例多样性则取决于具体脚本.
        \end{itemize}
        综合对比以上测试方法, 可以看出各种方法各有优劣. 其中, 本文提出的基于场景的测试方法的主要劣势在于自动化程度尚有一定不足, 需手工构建场景模型. 除此之外, 在测试能力方面则十分理想, 几乎接近自由度最高的测试脚本方法, 相比其他自动化方法有明显优势.
    
    \section{总结}
        本文提出了一种对web API的使用建模的场景模型, 以及基于该模型进行web API自动测试的方法, 并提供了该方法的工具原型实现, 最后, 通过实验评估了模型和方法的效果.
        
        场景模型基于概率有限状态自动机, 为了将之与API服务端点进行关联, 并表达数据流和控制流的约束、依赖与断言等, 本文对自动机模型进行了扩展, 并给出了扩展后的场景模型的形式化定义. 基于场景模型, 本文提出了自动测试的方法, 本文将包含请求数据的API调用/执行序列作为测试用例的模型, 详细阐述了如何从场景模型进行自动化的请求数据生成与执行序列生成, 并讨论了进一步扩展至启发式执行序列生成的初步思想. 然后, 本文在工具原型中实现了这一基于场景模型的测试方法, 工具原型OpenAPI格式脚本作为API行为描述文档, 使用YAML脚本作为场景模型文档, 与测试配置一起作为输入, 自动化测试用例的生成与执行结果作为输出. 使用工具, 本文在三个真实API服务上进行了实验, 并从四个方面评估了方法的表现与性能. 实验结果表明, 本文提出的测试方法可以生成具有多样性与高覆盖率的测试用例, 显著提高测试效率, 并发现真实系统中的故障, 具有实际使用价值.
    
    \section{展望}
    
        本工作还可以从以下几个方面进一步改进和扩展:
        \begin{itemize}
            \item 提升自动化程度: 一方面, 可以开发GUI工具将人工书写场景模型脚本改进为以拖拽和绘制方式搭建场景模型; 另一方面, 可以进一步探究从日志记录中挖掘使用场景, 自动生成场景模型的方法.
            \item 启发式测试序列生成: 目前的测试序列生成算法基于场景模型给出的概率分布, 初步的启发式算法尝试表现效果并不理想. 可以进一步尝试更复杂的启发式算法, 进行更智能的测试序列生成, 以进一步提高测试用例的覆盖率, 降低测试用例的冗余度.
            \item 大规模实验验证: 为提高本文实验结论的说服力, 可在更大规模的实际API应用中进行进一步的评估, 或者引入白盒指标如代码覆盖率和分支覆盖率, 来评估生成的测试用例指标.
        \end{itemize}
        
        笔者期待着, 在未来, 本文提出的场景模型和测试方法, 可以完善成为完整的工具链与生态链, 得到工业界的广泛运用, 切实提高软件测试的效率和质量.
