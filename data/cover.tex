\thusetup{
  %******************************
  % 注意:
  %   1. 配置里面不要出现空行
  %   2. 不需要的配置信息可以删除
  %******************************
  %
  %=====
  % 秘级
  %=====
  % secretlevel={秘密},
  % secretyear={10},
  %
  %=========
  % 中文信息
  %=========
  ctitle={Web API场景建模及
\\测试自动生成的研究与实现},
  cdegree={工学学士},
  cdepartment={计算机科学与技术系},
  cmajor={计算机科学与技术},
  cauthor={李林翼},
  csupervisor={白晓颖副教授},
  % cassosupervisor={陈文光教授}, % 副指导老师
  % ccosupervisor={某某某教授}, % 联合指导老师
  % 日期自动使用当前时间,若需指定按如下方式修改:
  cdate={2018年6月10日},
  %
  % 博士后专有部分
  % cfirstdiscipline={计算机科学与技术},
  % cseconddiscipline={系统结构},
  % postdoctordate={2009年7月——2011年7月},
  % id={编号}, % 可以留空: id={},
  % udc={UDC}, % 可以留空
  % catalognumber={分类号}, % 可以留空
  %
  %=========
  % 英文信息
  %=========
  etitle={Research and Implementation of Web API Scenario Modeling and Automated Test Generation},
  % 这块比较复杂,需要分情况讨论:
  % 1. 学术型硕士
  %    edegree:必须为Master of Arts或Master of Science(注意大小写)
  %             “哲学、文学、历史学、法学、教育学、艺术学门类,公共管理学科
  %              填写Master of Arts,其它填写Master of Science”
  %    emajor:“获得一级学科授权的学科填写一级学科名称,其它填写二级学科名称”
  % 2. 专业型硕士
  %    edegree:“填写专业学位英文名称全称”
  %    emajor:“工程硕士填写工程领域,其它专业学位不填写此项”
  % 3. 学术型博士
  %    edegree:Doctor of Philosophy(注意大小写)
  %    emajor:“获得一级学科授权的学科填写一级学科名称,其它填写二级学科名称”
  % 4. 专业型博士
  %    edegree:“填写专业学位英文名称全称”
  %    emajor:不填写此项
  edegree={Bachelor of Engineering},
  emajor={Computer Science and Technology},
  eauthor={Linyi Li},
  esupervisor={Associate Professor Xiaoying Bai},
  % eassosupervisor={Chen Wenguang},
  % 日期自动生成,若需指定按如下方式修改:
  % edate={December, 2005}
  %
  % 关键词用“英文逗号”分割
  ckeywords={Web API, 场景建模, 自动化测试, 软件测试, OpenAPI},
  ekeywords={Web API, Scenario Modeling, Automated Testing, Software Testing, OpenAPI}
}

% 定义中英文摘要和关键字
\begin{cabstract}
Web API在互联网软件中正被广泛应用. Web API可作为公开服务提供, 并集成于应用中, 满足各种用户需求. 因此, 作为应用的基础构件, web API的测试十分必要. 然而, 由于web API调用方式多样, 且常常动态更新与修改, 故其测试需要使用大量测试用例, 并需要定期回归测试, 因此, 手动测试具有很高代价.

本文提出了一种基于场景模型的web API自动化测试方法: 首先, 基于概率有限状态自动机进行扩展, 对API的使用进行了场景建模, 本文形式化定义的场景模型可以有效描述API数据流与控制流的约束、断言和状态转移, 并反映API的使用模式与频率. 然后提出了基于场景模型自动化生成测试用例的方法, 其包括请求数据与调用序列的自动化生成两方面. 最后, 本文介绍了Lapis工具, 它提供了该方法的完整实现. 利用该工具, 本文在实际web API服务上对模型和方法进行了评估, 从多个方面验证了其有效性和使用价值. 最后, 本文总结并展望了基于此场景模型进一步提高测试自动化程度和测试生成质量的研究方向.

  
\end{cabstract}

% 如果习惯关键字跟在摘要文字后面,可以用直接命令来设置,如下:
% \ckeywords{\TeX, \LaTeX, CJK, 模板, 论文}

\begin{eabstract}
Web API has been widely used for Internet software. Once published as open services, web APIs can be integrated into various applications to accommodate diversified usage scenarios. Therefore, as fundamental components, web API testing is inevitable. However, due to diversity of web API calling convention and high frequency of updates and modifications, web API testing requires a large number of test cases and frequent regression testing. Therefore, manual testing is expensive.

This thesis proposed a scenario-based approach for web API testing. Firstly, it extended probabilistic finite-state automata to model API usage. The formally defined scenario model facilitates the expression of API control flow and data flow constraints as well as assertions and state transitions. Meanwhile, the model could reveal the API usage pattern and usage frequency. Secondly, it proposed an approach to generate test cases from scenario model automatically, which includes request data automated generation and calling sequence automated generation. Lastly, it introduced the tool Lapis, which provides a full implementation of the approach. With the tool, we evaluated scenario model and approach on real web API services. The performance revealed the validity and valuable of them in several aspects. The thesis concluded with a brief summary and a discussion on how to further improve testing automaticity and quality based on proposed scenario model.
\end{eabstract}

% \ekeywords{\TeX, \LaTeX, CJK, template, thesis}
