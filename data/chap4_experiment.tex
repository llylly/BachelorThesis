\chapter{实验与评估}

    本工作的原型系统在三个实际的web API服务上进行了实验, 从四个方面评估了本文提出的场景模型和测试方法: 测试用例多样性, 测试覆盖率, 故障检测能力, 以及测试效率.
    
    \section{实验配置}
        实验选择的第一个被测系统为合作研究组开发的云对象存储服务(OSS)的系统原型(后文简称\textbf{OSS}). 此服务提供了33个web API接口, 并有较完整的使用文档. 测试时, 期望所有的API和功能点都业已实现完成.
        
        最近以来, 云对象存储服务(OSS)已经成为了流行的云存储服务形式. 许多大型云服务提供商均已提供云对象存储服务, 如亚马逊AWS\footnote{https://aws.amazon.com/s3}, IBM云\footnote{ https://console.bluemix.net/catalog/services/cloud-object-storage}和阿里云\footnote{ https://www.alibabacloud.com/product/oss}等. 在云对象存储服务中, 每个账号, 在每个服务器区域, 都可以有许多桶(bucket). 桶(bucket)与文件系统中的文件夹类似, 对象(object)与文件系统中的文件类似. 用户可以上传对象, 删除对象, 重命名对象, 创建符号链接等等. 每个对象都属于一个固定的桶.
        
        对于云对象存储服务, 本工作首先为这33个API编写了OpenAPI格式的API行为描述脚本. 然后, 设计了一些小型场景模型来检验每个API的基本功能. 在此之后, 设计了三个较大型的综合场景模型Scenario A、Scenario B、Scenario C以进行综合测试, 这三个场景均与实际使用场景较相似, 测试用例的生成与执行期望能够模拟实际部署后的高负载环境. 对于每个场景, 运行工具原型随机生成并执行了1,000个测试用例. 所有测试用例及其执行结果都被妥善保存. 后续分析时, 在所有测试用例上进行统计, 而没有进行任何遴选.
        
        第二个被测系统为阿里云的云服务器(ECS)服务(后文简称\textbf{ECS}), 阿里云是世界领先的云计算服务商, 淘宝、12306网站等大型应用均依托阿里云提供计算支持. 
        
        本实验测试阿里云服务为普通用户提供的云计算实例租用服务, 相关接口共有26个, 均为web API, 并配有专业使用文档. 在此服务中, 用户可以租用示例, 退租示例, 获取实例运行状态, 启动示例, 关闭示例, 重启示例等等. 我们选择了其中8个web API, 编写了行为描述脚本, 然后设计了包含所有这8个web API的综合场景模型进行测试. 一共随机生成并执行了100个测试用例, 所有测试用例在进行统计与分析时均纳入考虑.
        
        第三个被测系统是工业界合作团队提供的电子支付服务(后文简称\textbf{E-payment}). 本实验测试它的实时贷记API接口. 此API仅用于系统内部使用, 是微服务之间的通信接口, 因此使用LAN上的定制协议交互. 作为电子支付接口, 此服务拥有很高的可用率与很低的容错率要求. 同时, 它有多达20个参数. 由于模拟环境的缺失, 在本实验中, 暂未进行实际的调用与发送, 而是根据接口描述文档进行测试用例和请求数据的生成, 来验证生成的测试用例可以有效覆盖参数组合, 从而验证其在这类服务的测试上的可用性.
        
        对于E-payment服务, 实验中生成了巨量(超过两百万个)的测试用例, 并计算了这些测试用例对于参数组合的覆盖率. 这种方式不仅仅评估了测试用例的覆盖率, 也检验了工具原型的鲁棒性和性能.
        
    
    \section{测试用例多样性}
    
    \section{测试覆盖率}
    
    \section{故障检测能力}
    
    \section{测试效率}

