\chapter{场景建模与测试生成方法}
    \section{场景模型定义}
        使用场景模型是一种有效的模型化技术. 使用场景模型可以描述在现实世界中, 一个或多个人如何与系统进行交互\footnote{http://agilemodeling.com/artifacts/usageScenario.htm}. 模型包括了交互过程中的步骤, 事件和动作. 本文提出了一种描述与web API的交互的使用场景模型. 为了基于模型进行测试生成, 模型着重于抽象以下web API的特点:
        
        \begin{itemize}
            \item 与使用场景相关的服务的元素. 这包括服务的功能, 请求数据, 响应数据, 执行约束等.
            
            \item 交互中, 各服务之间的交互. 这包括执行序列和序列内通信, 比如调用序列中之前某个服务的响应, 可能作为之后调用其他服务的参数.
            
            \item 不同交互操作的频率. 在有限测试资源的情况下, 对于测试资源的分配, 各个被测API需要有优先级顺序. 比如, 一些API可能在许多场景下都被用到, 被频繁地调用, 或者为许多其他API提供前置服务. 那么, 根据基于使用的测试原则, 这些频繁使用的API需要进行更多测试. 因此, 对API使用的分析与建模有助于优化对各个API和场景所生成的测试用例数量.
        \end{itemize}
        
        概率有限状态自动机(PFSA)可以有效描述相关服务, 执行序列和交互频率. 然而, 对于数据约束和服务通信, 它的能力有所不足. 我们的模型基于概率有限状态自动机进行扩展, 克服了这些不足.
        
        \subsection{背景: 概率有限状态自动机}
        
        \subsection{模型定义}
        
        \subsection{解释与说明}
        
        \subsection{示例}
    
    \section{测试生成方法}
    
    \section{*测试场景优化}

